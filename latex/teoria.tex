\documentclass[../main.tex]{subfiles}

\begin{document}
\section{Osservabili macroscopici}
Prima di iniziare a modellizzare il problema \`e necessario domandarsi quali siano gli osservabili macroscopici principali e come siano legati tra di loro.
La descrizione di un sistema a livello macroscopico risulta critica ai fini della descrizione della dinamica.
\subsection{Densit\`a}
La densit\`a \`e una variabile tipicamente fisica adotatta nella teoria del traffico.
La densit\`a $\rho$ rappresenta il numero di veicoli per unit\`a di lunghezza della strada.
Sia ora $\Delta x$ la lunghezza di una strada in cui sono presenti $n$ veicoli, allora ad un tempo generico $t$ si ha che
\begin{equation*}
    \rho(n,t,\Delta x)=\frac{n(t)}{\Delta x}
\end{equation*}
La densit\`a si esprime in veicoli al kilometro (veh/km).
Tipicamente per ogni corsia di una strada si ha una $\rho_{max}\sim 10^2$ veh/km.
Si osservi ora come moltiplicando e dividendo per un infinitesimo temporale $dt$ il denominatore divenga l'area dell'intervallo di misura $S$.
In particolare
\begin{equation}
    \rho(t,\Delta x, S)=\frac{n(t)dt}{\Delta x dt}=\frac{\mbox{tempo totale trascorso in }S}{S}
    \label{eq:rho_s}
\end{equation}

\subsection{Flusso}
Il flusso $\Phi$ rappresenta il numero $m$ di veicoli che attraversano un certo localit\`a $x$ in un intervallo di tempo $\Delta t$
\begin{equation}
    \Phi(m, x, \Delta t)=\frac{m}{\Delta t}
\end{equation}
Il flusso \`e espresso in veicoli all'ora (veh/h).
Considerando ora un intorno infinitesimo $dx$ di $x$ \`e possibile ricavare la dipendenza più generale
\begin{equation}
    \Phi(x, \Delta t, S)=\frac{mdx}{\Delta t dx}=\frac{\mbox{distanza totale percorsa dai veicoli in }S}{S}
    \label{eq:phi_s}
\end{equation}

\subsection{Velocit\`a media}
La velocit\`a media \`e definita come il rapporto tra il flusso e la densit\`a: si nota immediatamente come questa non dipenda dall'area dell'intervallo di misura.
Unendo le Eq. (\ref{eq:phi_s}) e (\ref{eq:rho_s}):
\begin{equation}
    \bar{v}(x, t, S)=\frac{\Phi(x, \Delta t, S)}{\rho(t,\Delta x, S)}
\end{equation}
La relazione fondamentale della Traffic Flow Theory \cite{H111} \`e riassumibile nella:
\begin{equation}
    \Phi=\rho\bar{v}
    \label{eq:fundamental}
\end{equation}

\section{Diagrammi fondamentali}
A causa della relazione fondamentale del traffico riportata in Eq. (\ref{eq:fundamental}) risulta chiaro come su tre osservabili analizzati si abbiano solamente due variabili indipendenti.
%%
%Inserire immagine diagrammi fondamentali
\begin{figure}[H]
\centering
\begin{tikzpicture}
    
\begin{groupplot}[group style={group size=2 by 2}]
\nextgroupplot[
    axis y line = left,
    axis x line = bottom,
    ymax = 1.1,
    xmax = 1.1,
    ytick={0.5, 1},
    yticklabels={$\bar{v}_c$,$\bar{v}_f$},
    xtick={0.5, 1},
    xticklabels={$\rho_c$,$\rho_j$}]
\addplot [thick, domain=0:1] {-x+1};
\nextgroupplot[
    axis y line = left,
    axis x line = bottom,
    ymax = 0.823,
    xmin = -1,
    xmax = 0.1,
    ytick={0, 1/1.41},
    yticklabels={$\bar{v}_c$,$\bar{v}_f$},
    xtick={0},
    xticklabels={$\Phi_c$}]
\addplot [samples=200, thick, domain=-1:0] {sqrt(-x/2)};
\addplot [samples=200, thick,  domain=-1:0] {-sqrt(-x/2)};
\nextgroupplot[
    axis y line = left,
    axis x line = bottom,
    ymax = 1.1,
    xmax = 1.1,
    ytick={1},
    yticklabels={$\Phi_c$},
    xtick={0, 1},
    xticklabels={$\rho_c$,$\rho_j$}]
\addplot [samples=200, thick, domain=-1:1] {-x^2+1};
\end{groupplot}

\end{tikzpicture}
\caption[Diagrammi fondamentali]{\emph{Diagrammi fondamentali.}}
\end{figure}
%%
In una situazione stazionaria (rete in equilibrio) \`e possibile descrivere il sistema graficamente con tre diagrammi: $\bar{v}-\Phi$, $\Phi-\rho$ e $\bar{v}-\rho$.
La prima formulazione di questi \`e stata effettuata da Greenshield sulla base di alcune misurazioni da lui eseguite

\end{document}