\documentclass[../main.tex]{subfiles}

\begin{document}
Nella sezione precedente si \`e costruito il modello ed \`e emerso come questo si basi su alcuni parametri di controllo.
Variando questi parametri \`e di fatto possibile ottenere risultati estremamente diversi tra di loro, permettendo lo svolgimento di pi\`u studi.
In particolare, verranno effettuate tre tipologie di studi dove il caricamento della rete viene effettuato omogeneamente (inserendo contemporaneamente lo stesso numero di veicoli su tutte le strade):
\begin{itemize}
    \item \emph{Carico costante}: il reticolo viene sottoposto ad un carico costante per un certo lasso di tempo;
    \item \emph{Carico piccato}: il reticolo viene sovraccaricato nei primi istanti, durante i quali nessun veicolo pu\`o uscire, in modo tale da creare una congestione. Successivamente i veicoli vengono lasciati liberi di uscire dalla rete per permetterne lo svuotamento e studiarne gli effetti;
    \item \emph{Carico periodico}: il reticolo viene caricato nel tempo con una funzione sinusoidale, volta a rappresentare le diverse fasi di carico a cui una normale rete stradale \`e soggetta nell'arco di una giornata;
\end{itemize}

\section{Reticolo omogeneo}
Si consideri ora un network composto da 120 nodi disposti in un reticolo 10x12.
Su questo, si definiscano 16 differenti classi di veicoli, le cui sorgenti sono i nodi 1, 4, 7, 10 e le cui destinazioni sono i nodi 109, 112, 115, 118.
Risulta evidente dalla FIGURA CHE MANCA come i veicoli scorrano da un lato all'altro del reticolo.
Ogni strada abbia lunghezza e velocit\`a massima fissate a $500$ m e $50$ km/h, rispettivamente.
Si ponga ora la lunghezza media di un veicolo pari a 8 m, la velocit\`a minima su una strada pari a $v_{min}=\frac{3}{4}v_{max}$ e una probabilit\`a di errore dell'8\%.

\subsection{Carico costante}
Si vuole, come prima cosa, sottoporre il sistema ad un carico costante.
In questo modo si dovrebbe evitare la formazione di congestioni, lasciando al sistema la capacit\`a di trasporto.
Il sistema viene quindi sottoposto ad un carico costante di 250 ogni 60 s, fino a un tempo di  3.33 h, per un tempo totale di 4.15 h, lasciandoli liberi di uscire dalla rete una volta giunti a destinazione.
Ci si aspetta cos\`i una crescita iniziale della densit\`a media, la quale si mantiene costante fintantoch\'e nuovi veicoli vengono immessi nel sistema.
L'obiettivo \`e, infatti, avere tanti veicoli che escono quanti ne entrano.
\begin{figure}[H]
    \centering
    \begin{tikzpicture}
    \begin{axis}[
    grid = both,
    major grid style = {lightgray},
    minor grid style = {lightgray!25},
    width = 0.75\textwidth,
    height = 0.5\textwidth,
    ylabel near ticks,
    xlabel near ticks,
    xlabel = {Tempo (h)},
    ylabel = {Densit\`a media (veh/km)},]
    \addplot[
    mark size=0.69,
    draw=black,
    mark=o] file {./data/constant_homo/k-t.dat};
    \end{axis}
    \end{tikzpicture}
    \caption[Densit\`a media per un reticolo omogeneo con carico costante]{\emph{Densit\`a media per un reticolo omogeneo con carico costante.}}
    \label{fig:density_constant_homo}
\end{figure}
In Fig. \ref{fig:density_constant_homo} si pu\`o notare l'innalzamento iniziale, fino a circa 1.5 h, poi un regime pressoch\'e costante fino a 3.33 h, seguito da una rapida discesa.
Nell'arco temporale in cui la densit\`a \`e stabile, ossia da 1.5 h a 3.3 h, ci si aspetta un regime di traffico libero.
L'ipotesi \`e verificata in Fig. \ref{fig:nStreet_density_constant_homo} dove, graficando il numero di strade in funzione della densit\`a a 2 h, si pu\`o notare come la quasi totalit\`a di esse abbia densit\`a minima.
\begin{figure}[H]
    \centering
    \begin{tikzpicture}[scale=0.25]
    \begin{axis}[ybar interval,
    area style,
    width = 2\textwidth,
    height = 1.5\textwidth,
    xlabel = {Densit\`a di veicoli},
    ylabel = {Numero di strade},]
    \addplot+[
    ybar interval,
    mark=no,
    line width = 1.25pt
    ] plot file {./data/constant_homo/7200.dat};
    \end{axis}
    \end{tikzpicture}
    \caption[Distribuzione strade non congestionate per un reticolo omogeneo.]{\emph{Distribuzione strade non congestionate per un reticolo omogeneo.}}
    \label{fig:nStreet_density_constant_homo}
\end{figure}

\subsection{Carico piccato}
Si inserisca ora nel sistema un totale di 14000 veicoli, suddivisi in 1400 veicoli ogni 50 s, impedendone la fuoriuscita dal reticolo.
Se un veicolo arriva a destinazione, infatti, viene distrutto e ricreato in un nodo sorgente.
Si osserva in Fig. \ref{fig:density_peaked_homo} un picco iniziale della densit\`a media: questo rappresenta il sovraccarico del sistema.
\begin{figure}[H]
    \centering
    \begin{tikzpicture}
    \begin{axis}[
    grid = both,
    major grid style = {lightgray},
    minor grid style = {lightgray!25},
    width = 0.75\textwidth,
    height = 0.5\textwidth,
    ylabel near ticks,
    xlabel near ticks,
    xlabel = {Tempo (h)},
    ylabel = {Densit\`a media (veh/km)},]
    \addplot[
    mark size=0.69,
    draw=black,
    mark=o] file {./data/peaked_homo/k-t.dat};
    \end{axis}
    \end{tikzpicture}
    \caption[Densit\`a media per un reticolo omogeneo sovraccaricato]{\emph{Densit\`a media per un reticolo omogeneo sovraccaricato.}}
    \label{fig:density_peaked_homo}
\end{figure}
Una volta caricato il sistema ci si aspetta che la distribuzione di veicoli nelle strade non resti omogenea, anche a numero di veicoli costante.
Mantenendo il numero di agenti constante nel tempo, questi tenderanno ad occupare le strade costituenti i loro \emph{best path} e a lasciare vuote quelle pi\`u lontane da essi.
\begin{figure}[H]
    \centering
    \begin{tikzpicture}[scale=0.25]
    \begin{axis}[ybar interval,
    area style,
    width = 2\textwidth,
    height = 1.5\textwidth,
    xlabel = {Densità (veh/km)},
    ylabel = {Numero di strade},]
    \addplot+[
    ybar interval,
    mark=no,
    line width = 1.25pt
    ] plot file {./data/peaked_homo/540.dat};
    \end{axis}
    \end{tikzpicture}
    \caption{\emph{Numero di strade in funzione della densit\`a di veicoli.}}
    \label{fig:nStreet_density_peaked_homo}
\end{figure}
In Fig. \ref{fig:nStreet_density_peaked_homo} \`e visibile il diagramma caratteristico della congestione, in cui sono presenti, come atteso, un gran numero di strade vuote e altrettante sature.
In questo istante di tempo, pari a 540 s da inizio simulazione, ci si aspetta dunque che i diagrammi fondamentali siano confrontabili con quelli in Fig. \ref{fig:greenshield}: questa ipotesi \`e confermata dalla Fig. \ref{fig:DMF_peaked_homo}.
\begin{figure}[H]
    \begin{tikzpicture}[scale=0.6]
    \begin{axis}[
    grid = both,
    major grid style = {lightgray},
    minor grid style = {lightgray!25},
    width = 0.75\textwidth,
    height = 0.5\textwidth,
    ylabel near ticks,
    xlabel near ticks,
    xlabel = {Flusso (veh/h)},
    ylabel = {Velocit\`a media (km/h)},]
    \addplot[
    mark size=0.69,
    draw=black,
    only marks] file {./data/peaked_homo/540_u-q.dat};
    \end{axis}
    \end{tikzpicture}\hfill
    \begin{tikzpicture}[scale=0.6]
        \begin{axis}[
        grid = both,
        major grid style = {lightgray},
        minor grid style = {lightgray!25},
        width = 0.75\textwidth,
        height = 0.5\textwidth,
        ylabel near ticks,
        xlabel near ticks,
        xlabel = {Densità (veh/km)},
        ylabel = {Flusso (veh/h)},]
        \addplot[
        mark size=0.69,
        draw=black,
        only marks] file {./data/peaked_homo/540_q-k.dat};
        \end{axis}
    \end{tikzpicture}
    \centering
    \begin{tikzpicture}[scale=0.6]
    \begin{axis}[
    grid = both,
    major grid style = {lightgray},
    minor grid style = {lightgray!25},
    width = 0.75\textwidth,
    height = 0.5\textwidth,
    ylabel near ticks,
    xlabel near ticks,
    xlabel = {Densit\`a (veh/km)},
    ylabel = {Velocit\`a media (km/h)},]
    \addplot[
    mark size=0.69,
    draw=black,
    only marks] file {./data/peaked_homo/540_u-k.dat};
    \end{axis}
    \end{tikzpicture}
    \caption[DFM per una congestione]{\emph{Diagrammi fondamentali macroscopici di una congestione in un reticolo omogeneo.}}
    \label{fig:DMF_peaked_homo}
\end{figure}
Secondo la teoria dei flussi di traffico, un sistema congestionato tende a risolvere la congestione in modo differente da come questa si \`e formata: questo fenomeno causa la formazione di un ciclo di isteresi, visibile in Fig. \ref{fig:hysteresys_peaked_homo}.
\begin{figure}[H]
    \begin{tikzpicture}[scale=0.6]
        \begin{axis}[
        grid = both,
        major grid style = {lightgray},
        minor grid style = {lightgray!25},
        width = 0.75\textwidth,
        height = 0.5\textwidth,
        ylabel near ticks,
        xlabel near ticks,
        xlabel = {Densità (veh/km)},
        ylabel = {Flusso (veh/h)},]
        \addplot[
        mark size=0.69,
        draw=black,
        mark=o] file {./data/peaked_homo/q-k.dat};
        \end{axis}
    \end{tikzpicture}\hfill
    \begin{tikzpicture}[scale=0.6]
    \begin{axis}[
    grid = both,
    major grid style = {lightgray},
    minor grid style = {lightgray!25},
    width = 0.75\textwidth,
    height = 0.5\textwidth,
    ylabel near ticks,
    xlabel near ticks,
    xlabel = {Densit\`a (veh/km)},
    ylabel = {Velocit\`a media (km/h)},]
    \addplot[
    mark size=0.69,
    draw=black,
    mark=o] file {./data/peaked_homo/u-k.dat};
    \end{axis}
    \end{tikzpicture}
    \caption[Isteresi per un reticolo omogeneo sovraccaricato]{\emph{Ciclo di isteresi per un reticolo omogeneo sovraccaricato.}}
    \label{fig:hysteresys_peaked_homo}
\end{figure}
Si pu\`o notare come sia il flusso che la densit\`a crescano fino a un punto critico nel quale si ha un calo drastico del flusso dovuto all'eccessiva presenza di veicoli sulle strade.
Il sistema risulta qui congestionato e perde la capacit\`a di trasporto.
Permettendo ai veicoli di uscire senza rientrare, si inizia a scaricare il sistema: lo scarico, che avviene gradualmente, prevede un ritorno alla stabilit\`a lungo una traiettoria differente rispetto alla fase di carico.
Tale differenza \`e visibile sia sul piano $\Phi/\rho$, sia sul piano $\bar{v}/\rho$.

\subsection{Carico periodico}
Si vuole ora sottoporre il sistema ad un carico realistico, quindi variabile nell'arco di una giornata.
Per semplicit\`a si decide di trascurare il carico notturno, in quanto nella maggior parte dei casi lascia il sistema in uno stato di flusso libero.
Durante il giorno, invece, dalle acquisizioni reali (MANCA REF) si notano diversi picchi: il primo al mattino verso le ore 9:00, il secondo all'ora di pranzo, verso le 13:00 e il terzo al rientro serale, alle 17:00.
Gli orari dei picchi sono stati leggermente approssimati in modo tale da semplificare la funzione in ingresso, la quale deve avere un picco ogni 4 ore.
Assumendo come orario di inizio simulazione le 5:45 del mattino, si immettano veicoli uniformemente nel sistema secondo la seguente funzione:
\begin{equation}
    \Delta n(t) = A \left\lvert \sin\left(\frac{2\pi}{T}t\right) \right\rvert 
\end{equation}
con $A = 2200$ veh e $T = 32400$ s.
In questo caso, non vi \`e accumulo di veicoli: in ogni istante i veicoli che giungono a destinazione vengono eliminati.
Ci si aspetta in questo modo che, avendo la funzione tre picchi, il sistema compia tre cicli di caricamento e scaricamento.
\begin{figure}[H]
    \centering
    \begin{tikzpicture}
    \begin{axis}[
    grid = both,
    major grid style = {lightgray},
    minor grid style = {lightgray!25},
    width = 0.75\textwidth,
    height = 0.5\textwidth,
    ylabel near ticks,
    xlabel near ticks,
    xlabel = {Tempo (h)},
    ylabel = {Densit\`a media (veh/km)},]
    \addplot[
    mark size=0.69,
    draw=black,
    mark=o] file {./data/periodic_homo/k-t.dat};
    \end{axis}
    \end{tikzpicture}
    \caption[Variazione periodica della densit\`a in un reticolo omogeneo]{\emph{Variazione della densit\`a media delle strade della rete nel tempo. Si possono notare tre picchi: il primo alle 8:45, il secondo alle 13:00 e il terzo alle 17:45.}}
    \label{fig:density_time_periodic_homo}
\end{figure}
I tre picchi sono ben visibili in Fig. \ref{fig:density_time_periodic_homo}.
A ogni picco corrisponde una fase di carico/scarico, quindi, come osservato precedentemente in Fig. \ref{fig:hysteresys_peaked_homo}, ci si aspetta la presenza di tre cicli di isteresi.
\begin{figure}[H]
    \centering
    \begin{tikzpicture}
    \begin{axis}[
    grid = both,
    major grid style = {lightgray},
    minor grid style = {lightgray!25},
    width = 0.75\textwidth,
    height = 0.5\textwidth,
    ylabel near ticks,
    xlabel near ticks,
    xlabel = {Densit\`a media (veh/km)},
    ylabel = {Flusso medio (veh/h)},]
    \addplot[
    mark size=0.69,
    draw=black,
    mark=o] file {./data/periodic_homo/q-k.dat};
    \end{axis}
    \end{tikzpicture}
    \caption[Isteresi con carico periodico e reticolo omogeneo]{\emph{Cicli di isteresi sul piano $\Phi/\rho$ per un reticolo omogeneo caricato periodicamente.}}
    \label{fig:hysteresys_periodic_homo}
\end{figure}
Come si pu\`o notare in Fig. \ref{fig:hysteresys_periodic_homo} sono effettivamente presenti tre diversi cicli di isteresi.
Si osserva, inoltre, che il secondo ciclo ha ampiezza inferiore, essendo il picco di densit\`a meno pronunciato.

\section{Rete stradale di Rimini (?)}

\section{Discussione}

% \begin{figure}[H]
%     \centering
%     \begin{tikzpicture}[scale=0.5]
%     \begin{axis}[ybar interval,
%     area style,
%     width = 2\textwidth,
%     height = 1.5\textwidth,
%     xlabel = {Densità di veicoli},
%     ylabel = {Numero di strade},]
%     \addplot+[
%     ybar interval,
%     mark=no,
%     line width = 1.25pt
%     ] plot file {./temp_data/250.dat};
%     \end{axis}
%     \end{tikzpicture}
%     \caption{\emph{Prova.}}
% \end{figure}

% \begin{figure}[H]
%     \begin{tikzpicture}[scale=0.5]
%     \begin{axis}[ybar interval,
%     area style,
%     width = 2\textwidth,
%     height = 1.5\textwidth,
%     xlabel = {Densità di veicoli},
%     ylabel = {Numero di strade},]
%     \addplot+[
%     ybar interval,
%     mark=no,
%     line width = 1.25pt
%     ] plot file {./temp_data/1250.dat};
%     \end{axis}
%     \end{tikzpicture}
%     \caption{\emph{Prova.}}
% \end{figure}

\begin{figure}[H]
    \centering
    \begin{tikzpicture}[scale=0.5]
    \begin{axis}[ybar interval,
    area style,
    width = 2\textwidth,
    height = 1.5\textwidth,
    xlabel = {Densità di veicoli},
    ylabel = {Numero di strade},]
    \addplot+[
    ybar interval,
    mark=no,
    line width = 1.25pt
    ] plot file {./temp_data/7200.dat};
    \end{axis}
    \end{tikzpicture}
    \caption{\emph{Prova.}}
\end{figure}

\begin{figure}[H]
    \begin{tikzpicture}[scale=0.6]
    \begin{axis}[
    grid = both,
    major grid style = {lightgray},
    minor grid style = {lightgray!25},
    width = 0.75\textwidth,
    height = 0.5\textwidth,
    ylabel near ticks,
    xlabel near ticks,
    xlabel = {Flusso (veh/h)},
    ylabel = {Velocit\`a media (km/h)},]
    \addplot[
    mark size=0.69,
    draw=black,
    only marks] file {./temp_data/7200_u-q.dat};
    \end{axis}
    \end{tikzpicture}\hfill
    \begin{tikzpicture}[scale=0.6]
        \begin{axis}[
        grid = both,
        major grid style = {lightgray},
        minor grid style = {lightgray!25},
        width = 0.75\textwidth,
        height = 0.5\textwidth,
        ylabel near ticks,
        xlabel near ticks,
        xlabel = {Densità (veh/km)},
        ylabel = {Flusso (veh/h)},]
        \addplot[
        mark size=0.69,
        draw=black,
        only marks] file {./temp_data/7200_q-k.dat};
        \end{axis}
    \end{tikzpicture}
    \centering
    \begin{tikzpicture}[scale=0.6]
    \begin{axis}[
    grid = both,
    major grid style = {lightgray},
    minor grid style = {lightgray!25},
    width = 0.75\textwidth,
    height = 0.5\textwidth,
    ylabel near ticks,
    xlabel near ticks,
    xlabel = {Densit\`a (veh/km)},
    ylabel = {Velocit\`a media (km/h)},]
    \addplot[
    mark size=0.69,
    draw=black,
    only marks] file {./temp_data/7200_u-k.dat};
    \end{axis}
    \end{tikzpicture}
    \caption[Diagrammi fondamentali con distribuzione omogenea]{\emph{Diagrammi fondamentali con distribuzione omogenea.}}
\end{figure}

\begin{figure}[H]
    \centering
    \begin{tikzpicture}
    \begin{axis}[
    grid = both,
    major grid style = {lightgray},
    minor grid style = {lightgray!25},
    width = 0.75\textwidth,
    height = 0.5\textwidth,
    ylabel near ticks,
    xlabel near ticks,
    xlabel = {Densit\`a media (veh/km)},
    ylabel = {Flusso medio (veh/h)},]
    \addplot[
    mark size=0.69,
    draw=black,
    mark=o] file {./temp_data/q-k.dat};
    \end{axis}
    \end{tikzpicture}
\end{figure}

\begin{figure}[H]
    \centering
    \begin{tikzpicture}
    \begin{axis}[
    grid = both,
    major grid style = {lightgray},
    minor grid style = {lightgray!25},
    width = 0.75\textwidth,
    height = 0.5\textwidth,
    ylabel near ticks,
    xlabel near ticks,
    xlabel = {Densit\`a media (veh/km)},
    ylabel = {Velocit\`a medio (veh/h)},]
    \addplot[
    mark size=0.69,
    draw=black,
    mark=o] file {./temp_data/u-k.dat};
    \end{axis}
    \end{tikzpicture}
\end{figure}

\begin{figure}[H]
    \centering
    \begin{tikzpicture}
    \begin{axis}[
    grid = both,
    major grid style = {lightgray},
    minor grid style = {lightgray!25},
    width = 0.75\textwidth,
    height = 0.5\textwidth,
    ylabel near ticks,
    xlabel near ticks,
    xlabel = {Tempo (h)},
    ylabel = {Densit\`a media (veh/km)},]
    \addplot[
    mark size=0.69,
    draw=black,
    mark=o] file {./temp_data/k-t.dat};
    \end{axis}
    \end{tikzpicture}
\end{figure}

\begin{figure}[H]
    \centering
    \begin{tikzpicture}
    \begin{axis}[
    grid = both,
    major grid style = {lightgray},
    minor grid style = {lightgray!25},
    width = 0.75\textwidth,
    height = 0.5\textwidth,
    ylabel near ticks,
    xlabel near ticks,
    xlabel = {Tempo (h)},
    ylabel = {Flusso medio (veh/h)},]
    \addplot[
    mark size=0.69,
    draw=black,
    mark=o] file {./temp_data/q-t.dat};
    \end{axis}
    \end{tikzpicture}
\end{figure}

\begin{figure}[H]
    \centering
    \begin{tikzpicture}
    \begin{axis}[
    grid = both,
    major grid style = {lightgray},
    minor grid style = {lightgray!25},
    width = 0.75\textwidth,
    height = 0.5\textwidth,
    ylabel near ticks,
    xlabel near ticks,
    xlabel = {Tempo (h)},
    ylabel = {Velocit\`a media (km/h)},]
    \addplot[
    mark size=0.69,
    draw=black,
    mark=o] file {./temp_data/u-t.dat};
    \end{axis}
    \end{tikzpicture}
\end{figure}
\end{document}