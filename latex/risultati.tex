\documentclass[../main.tex]{subfiles}

\begin{document}

\section{Modalit\`a di esecuzione}
Nella sezione precedente si \`e costruito il modello ed \`e emerso come questo si basi su alcuni parametri di controllo.
Variando questi parametri \`e di fatto possibile ottenere risultati estremamente diversi tra di loro.
Assumendo che uno step temporale del modello sia equivalente a $1$ s 

\section{Diagrammi fondamentali}

\section{Isteresi}

\begin{figure}[H]
    \centering
    \begin{tikzpicture}[scale=0.5]
    \begin{axis}[ybar interval,
    area style,
    width = 2\textwidth,
    height = 1.5\textwidth,
    xlabel = {Densità di veicoli},
    ylabel = {Numero di strade},]
    \addplot+[
    ybar interval,
    mark=no,
    line width = 1.25pt
    ] plot file {./data/250.dat};
    \end{axis}
    \end{tikzpicture}
    \caption{\emph{Prova.}}
\end{figure}

\begin{figure}[H]
    \begin{tikzpicture}[scale=0.5]
    \begin{axis}[ybar interval,
    area style,
    width = 2\textwidth,
    height = 1.5\textwidth,
    xlabel = {Densità di veicoli},
    ylabel = {Numero di strade},]
    \addplot+[
    ybar interval,
    mark=no,
    line width = 1.25pt
    ] plot file {./data/1250.dat};
    \end{axis}
    \end{tikzpicture}
    \caption{\emph{Prova.}}
\end{figure}

\begin{figure}[H]
    \centering
    \begin{tikzpicture}[scale=0.5]
    \begin{axis}[ybar interval,
    area style,
    width = 2\textwidth,
    height = 1.5\textwidth,
    xlabel = {Densità di veicoli},
    ylabel = {Numero di strade},]
    \addplot+[
    ybar interval,
    mark=no,
    line width = 1.25pt
    ] plot file {./data/10000.dat};
    \end{axis}
    \end{tikzpicture}
    \caption{\emph{Prova.}}
\end{figure}

\begin{figure}[H]
    \begin{tikzpicture}[scale=0.6]
    \begin{axis}[
    grid = both,
    major grid style = {lightgray},
    minor grid style = {lightgray!25},
    width = 0.75\textwidth,
    height = 0.5\textwidth,
    ylabel near ticks,
    xlabel near ticks,
    xlabel = {Flusso (veh/h)},
    ylabel = {Velocit\`a media (km/h)},]
    \addplot[
    mark size=0.69,
    draw=black,
    only marks] file {./data/250_u-q.dat};
    \end{axis}
    \end{tikzpicture}\hfill
    \begin{tikzpicture}[scale=0.6]
        \begin{axis}[
        grid = both,
        major grid style = {lightgray},
        minor grid style = {lightgray!25},
        width = 0.75\textwidth,
        height = 0.5\textwidth,
        ylabel near ticks,
        xlabel near ticks,
        xlabel = {Densità (veh/km)},
        ylabel = {Flusso (veh/h)},]
        \addplot[
        mark size=0.69,
        draw=black,
        only marks] file {./data/250_q-k.dat};
        \end{axis}
    \end{tikzpicture}
    \centering
    \begin{tikzpicture}[scale=0.6]
    \begin{axis}[
    grid = both,
    major grid style = {lightgray},
    minor grid style = {lightgray!25},
    width = 0.75\textwidth,
    height = 0.5\textwidth,
    ylabel near ticks,
    xlabel near ticks,
    xlabel = {Densit\`a (veh/km)},
    ylabel = {Velocit\`a media (km/h)},]
    \addplot[
    mark size=0.69,
    draw=black,
    only marks] file {./data/250_u-k.dat};
    \end{axis}
    \end{tikzpicture}
    \caption[Diagrammi fondamentali con distribuzione omogenea]{\emph{Diagrammi fondamentali con distribuzione omogenea.}}
\end{figure}

\begin{figure}[H]
    \centering
    \begin{tikzpicture}
    \begin{axis}[
    grid = both,
    major grid style = {lightgray},
    minor grid style = {lightgray!25},
    width = 0.75\textwidth,
    height = 0.5\textwidth,
    ylabel near ticks,
    xlabel near ticks,
    xlabel = {Densit\`a (veh/km)},
    ylabel = {Flusso (veh/h)},]
    \addplot[
    mark size=0.69,
    draw=black,
    mark=o] file {./prova.dat};
    \end{axis}
    \end{tikzpicture}
\end{figure}
\end{document}