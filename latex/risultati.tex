\documentclass[../main.tex]{subfiles}

\begin{document}
Nella sezione precedente si \`e costruito il modello ed \`e emerso come questo si basi su alcuni parametri di controllo.
Variando questi parametri \`e di fatto possibile ottenere risultati estremamente diversi tra di loro, permettendo lo svolgimento di pi\`u studi.
In particolare, verranno effettuate tre tipologie di studi per diversi reticoli stradali:
\begin{itemize}
    \item \emph{Carico adiabatico omogeneo}: il reticolo viene caricato omogeneamente (inserendo contemporaneamente lo stesso numero di veicoli su tutte le strade), lentamente e per diverse volte, in modo tale da raggiungere un nuovo equilibrio prima dell'immissione di nuovi veicoli;
    \item \emph{Carico periodico}: il reticolo viene caricato nel tempo con una funzione sinusoidale, volta a rappresentare le diverse fasi di carico a cui una normale rete stradale \`e soggetta nell'arco di una giornata;
    \item \emph{Limite di random walk}: il reticolo viene sovraccaricato uniformemente. Vengono inoltre resi equiprobabili tutti i percorsi esistenti tra sorgente e destinazione.
\end{itemize}

\section{Reticolo omogeneo}
Si consideri ora un network composto da 120 nodi disposti in un reticolo 10x12.
Su questo, si definiscano 16 differenti classi di veicoli, le cui sorgenti sono i nodi 1, 4, 7, 10 e le cui destinazioni sono i nodi 109, 112, 115, 118.
Risulta evidente dalla FIGURA CHE MANCA come i veicoli scorrano da un lato all'altro del reticolo.
Ogni strada abbia lunghezza e velocit\`a massima fissate a $500$ m e $50$ km/h, rispettivamente.
Si ponga ora la lunghezza media di un veicolo pari a $4$ m, la velocit\`a minima su una strada pari a $v_{min}=\frac{3}{4}v_{max}$ e una probabilit\`a di errore del $25$\% (DA RIVEDERE).

\subsection{Carico adiabatico omogeneo}
Si inserisca ora nel sistema un totale di 20000 veicoli, suddivisi in 200 veicoli ogni $50$ s.

\begin{figure}[H]
    \begin{tikzpicture}[scale=0.6]
        \begin{axis}[
        grid = both,
        major grid style = {lightgray},
        minor grid style = {lightgray!25},
        width = 0.75\textwidth,
        height = 0.5\textwidth,
        ylabel near ticks,
        xlabel near ticks,
        xlabel = {Densità (veh/km)},
        ylabel = {Flusso (veh/h)},]
        \addplot[
        mark size=0.69,
        draw=black,
        only marks] file {./data/adiabatic_homo/750_q-k.dat};
        \end{axis}
    \end{tikzpicture}\hfill
    \begin{tikzpicture}[scale=0.6]
    \begin{axis}[
    grid = both,
    major grid style = {lightgray},
    minor grid style = {lightgray!25},
    width = 0.75\textwidth,
    height = 0.5\textwidth,
    ylabel near ticks,
    xlabel near ticks,
    xlabel = {Flusso (veh/h)},
    ylabel = {Velocit\`a media (km/h)},]
    \addplot[
    mark size=0.69,
    draw=black,
    only marks] file {./data/adiabatic_homo/750_u-q.dat};
    \end{axis}
    \end{tikzpicture}
    \centering
    \begin{tikzpicture}[scale=0.6]
    \begin{axis}[
    grid = both,
    major grid style = {lightgray},
    minor grid style = {lightgray!25},
    width = 0.75\textwidth,
    height = 0.5\textwidth,
    ylabel near ticks,
    xlabel near ticks,
    xlabel = {Densit\`a (veh/km)},
    ylabel = {Velocit\`a media (km/h)},]
    \addplot[
    mark size=0.69,
    draw=black,
    only marks] file {./data/adiabatic_homo/750_u-k.dat};
    \end{axis}
    \end{tikzpicture}
    \caption[DMF stazionari con carico adiabatico e reticolo omogeneo]{\emph{Diagrammi fondamentali macroscopici in situazione stazionaria per un reticolo omogeneo caricato adiabaticamente.}}
    \label{fig:FMD_adiabatic_homo}
\end{figure}

\subsection{Carico periodico}

\subsection{Limite di random walk}

\section{Rete stradale di Rimini (?)}

\subsection{Carico adiabatico omogeneo}

\subsection{Carico periodico}

\subsection{Limite di random walk}

\section{Discussione}

\begin{figure}[H]
    \centering
    \begin{tikzpicture}[scale=0.5]
    \begin{axis}[ybar interval,
    area style,
    width = 2\textwidth,
    height = 1.5\textwidth,
    xlabel = {Densità di veicoli},
    ylabel = {Numero di strade},]
    \addplot+[
    ybar interval,
    mark=no,
    line width = 1.25pt
    ] plot file {./temp_data/250.dat};
    \end{axis}
    \end{tikzpicture}
    \caption{\emph{Prova.}}
\end{figure}

\begin{figure}[H]
    \begin{tikzpicture}[scale=0.5]
    \begin{axis}[ybar interval,
    area style,
    width = 2\textwidth,
    height = 1.5\textwidth,
    xlabel = {Densità di veicoli},
    ylabel = {Numero di strade},]
    \addplot+[
    ybar interval,
    mark=no,
    line width = 1.25pt
    ] plot file {./temp_data/1250.dat};
    \end{axis}
    \end{tikzpicture}
    \caption{\emph{Prova.}}
\end{figure}

\begin{figure}[H]
    \centering
    \begin{tikzpicture}[scale=0.5]
    \begin{axis}[ybar interval,
    area style,
    width = 2\textwidth,
    height = 1.5\textwidth,
    xlabel = {Densità di veicoli},
    ylabel = {Numero di strade},]
    \addplot+[
    ybar interval,
    mark=no,
    line width = 1.25pt
    ] plot file {./temp_data/750.dat};
    \end{axis}
    \end{tikzpicture}
    \caption{\emph{Prova.}}
\end{figure}

\begin{figure}[H]
    \begin{tikzpicture}[scale=0.6]
    \begin{axis}[
    grid = both,
    major grid style = {lightgray},
    minor grid style = {lightgray!25},
    width = 0.75\textwidth,
    height = 0.5\textwidth,
    ylabel near ticks,
    xlabel near ticks,
    xlabel = {Flusso (veh/h)},
    ylabel = {Velocit\`a media (km/h)},]
    \addplot[
    mark size=0.69,
    draw=black,
    only marks] file {./temp_data/750_u-q.dat};
    \end{axis}
    \end{tikzpicture}\hfill
    \begin{tikzpicture}[scale=0.6]
        \begin{axis}[
        grid = both,
        major grid style = {lightgray},
        minor grid style = {lightgray!25},
        width = 0.75\textwidth,
        height = 0.5\textwidth,
        ylabel near ticks,
        xlabel near ticks,
        xlabel = {Densità (veh/km)},
        ylabel = {Flusso (veh/h)},]
        \addplot[
        mark size=0.69,
        draw=black,
        only marks] file {./temp_data/750_q-k.dat};
        \end{axis}
    \end{tikzpicture}
    \centering
    \begin{tikzpicture}[scale=0.6]
    \begin{axis}[
    grid = both,
    major grid style = {lightgray},
    minor grid style = {lightgray!25},
    width = 0.75\textwidth,
    height = 0.5\textwidth,
    ylabel near ticks,
    xlabel near ticks,
    xlabel = {Densit\`a (veh/km)},
    ylabel = {Velocit\`a media (km/h)},]
    \addplot[
    mark size=0.69,
    draw=black,
    only marks] file {./temp_data/750_u-k.dat};
    \end{axis}
    \end{tikzpicture}
    \caption[Diagrammi fondamentali con distribuzione omogenea]{\emph{Diagrammi fondamentali con distribuzione omogenea.}}
\end{figure}

\begin{figure}[H]
    \centering
    \begin{tikzpicture}
    \begin{axis}[
    grid = both,
    major grid style = {lightgray},
    minor grid style = {lightgray!25},
    width = 0.75\textwidth,
    height = 0.5\textwidth,
    ylabel near ticks,
    xlabel near ticks,
    xlabel = {Densit\`a media (veh/km)},
    ylabel = {Flusso medio (veh/h)},]
    \addplot[
    mark size=0.69,
    draw=black,
    mark=o] file {./temp_data/q-k.dat};
    \end{axis}
    \end{tikzpicture}
\end{figure}

\begin{figure}[H]
    \centering
    \begin{tikzpicture}
    \begin{axis}[
    grid = both,
    major grid style = {lightgray},
    minor grid style = {lightgray!25},
    width = 0.75\textwidth,
    height = 0.5\textwidth,
    ylabel near ticks,
    xlabel near ticks,
    xlabel = {Densit\`a media (veh/km)},
    ylabel = {Velocit\`a media (km/h)},]
    \addplot[
    mark size=0.69,
    draw=black,
    mark=o] file {./temp_data/u-k.dat};
    \end{axis}
    \end{tikzpicture}
\end{figure}
\end{document}