\documentclass[../main.tex]{subfiles}

\begin{document}
Pur sembrandone naturalmete sconnesso, il traffico risulta strettamente legato alla fisica.
\`E possibile, infatti, considerare ogni veicolo come una \emph{particella elementare} vincolata a muoversi su una traiettoria unidimensionale.
Questa particella deve ovviamente obbedire ad alcune regole: deve, ad esempio, spostarsi tra due punti $A$ e $B$ senza collidere con altre particelle.\\
Un modello di sistema complesso cos\`i definito \`e in grado di spiegare fisicamente fenomeni come le congestioni?

Negli ultimi 70 anni, gli scienziati hanno sviluppato diversi modelli e teorie sui flussi di traffico per comprendere questi fenomeni non lineari \cite{bs2004physics}.
I primi modelli sono stati sviluppati da Reushel (1950) e Pipes (1953), entrambi microscopici e rappresentanti il movimento di macchine in moto le une vicine alle altre su una strada a singola corsia.
Caratteristica in comune \`e l'assunzione che la velocit\`a di un veicolo dipenda linearmente sia dalla distanza dal veicolo precedente che dalla distanza dal successivo.
Nonostante l'ipotesi sembrasse ragionevole, la mancanza di conferme sperimentali sanc\`i il fallimento del modello.
Pochi anni dopo, nel 1955, Lighthill, un famoso teorico della meccanica dei fluidi, e Whitham proposero un modello macroscopico per i flussi di traffico, in analogia con il comportamento dei fluidi.
Le ipotesi alla base di questo modello sono la conservazione del numero di veicoli totali, ben giustificata, e l'esistenza di un'equazione di stato in grado di descrivere una relazione tra flusso di traffico (veh/h) e densit\`a (veh/km).
La seconda ipotesi, pur apparentemente ingiustificata, trov\`o presto conferme sperimentali.
Inoltre, il modello riusc\`i a spiegare fenomeni come le \emph{shock waves}, generate dal cambiamento del traffico verso uno stato con differenti densit\`a e flusso. 

% Fai qualche esempio dai, so che vuoi leggerti 200 paper diversi che parlano di traffico
Cos\`i come difetti e impurit\`a sono importanti per le transizioni di fase nei sistemi fisici, i \emph{bottleneck} (ingorghi) lo sono per i sistemi di traffico.
Le cause degli ingorghi sono molteplici e spesso legate alla struttura stessa della strada: alcuni esempi possono essere una improvvisa riduzione delle corsie, cantieri stradali, curve, ecc.



\end{document}