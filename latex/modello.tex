\documentclass[../main.tex]{subfiles}

\begin{document}
Si consideri una matrice di adiacenza pesata $\mathcal{W}_{ij}$ che differisce dalla matrice di adiacenza $\mathcal{A}_{ij}$ in quanto non possiede solamente i valori $\left\{0,1\right\}$.
In particolare, questi pesi rappresentino la lunghezza delle strade (collegamenti) tra i vari nodi della rete, ovvero $\mathcal{W}_{ij} \geq 0$.
Una volta definito il network stradale \`e possibile definire le classi di agenti $c(s,d)$ che vi circoleranno sopra, le quali saranno definite da un nodo $s$ sorgente e da un nodo $d$ destinazione.
Ogni individuo dovrà quindi muovere dalla sua sorgente alla destinazione prefissate, seguendo un percorso che si suppone sia il minimo tra tutti i percorsi disponibili.
\end{document}