\documentclass[../main.tex]{subfiles}

\begin{document}

\section*{Random Walk su network}
Si consideri ora un network generico con $M$ nodi, assumendo che ognuno di essi possa scambiare particelle coi suoi vicini.
Sia $\pi_{ij}$ la matrice stocastica che definisce la probabilità che una particella effettui il viaggio tra nodi $j\to i$.
Assumendo di avere $N$ particelle nel reticolo, è possibile definire la funzione $\delta_\alpha(i,t)$ che vale $1$ se la particella $\alpha$ si trova nel nodo $i$ al tempo $t$, 0 altrimenti.
\\Ogni particella segue quindi la dinamica
\begin{equation}
    \delta_\alpha(i,t+\Delta t)=\sum_j\xi_{ij}^\alpha\delta_\alpha(j,t)
\end{equation}
dove $\xi_{ij}^\alpha$ è una matrice random che prende valori della base standard $\widehat{e}_i\in \mathbb{R}^M$ con probabilità $\pi_{ij}$.
Il numero di particelle nel nodo $i$ al tempo $t$ è dato da
\begin{equation}
    n_i(t)=\sum_\alpha\delta_\alpha(i,t)
\end{equation}
ed è possibile dimostrare \cite{RandomWalks} che la seguente equazione è un integrale del moto
\begin{equation}
    \sum_in_i(t)=N
\end{equation}

\section*{Modelli di traffico}

\end{document}