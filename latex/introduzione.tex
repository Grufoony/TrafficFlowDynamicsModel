\documentclass[../main.tex]{subfiles}

\begin{document}
Le congestioni nel traffico sono ad oggi uno dei maggiori problemi per lo sviluppo delle citt\`a.
Nel solo stato della Florida, ad esempio, \`e stato stimato che dal 2003 al 2007 queste abbiano causato perdite dai 4.5 ai 7 miliardi di dollari annui \cite{florida}.
Nell'ultimo decennio, con l'obiettivo di attenuare questa problematica, nelle grandi citt\`a hanno cominciato a proliferare diverse compagnie di ride-hailing, come Uber.
Nonostante il beneficio economico portato da esse e dalla competitivit\`a del mercato la presenza in gran numero di questi fornitori di servizi va a peggiorare la qualit\`a della mobilit\`a urbana.
Questo effetto diventa abbastanza rilevante in zone nelle quali vi \`e poca domanda o la velocit\`a stradale media \`e sufficientemente bassa.
Ogni operatore aggiunto di questo settore causa, ad esempio, un aumento del numero totale di veicoli su strada del 2.5\% a Manhattan e del 37\% a San Francisco \cite{Kondor2022}.
\\Un ruolo fondamentale nella gestione delle congestioni nelle citt\`a lo gioca la logistica delle stesse.
Si consideri ora la rete stradale di una citt\`a generica di superfice approssimabile a quella di una circonferenza di raggio $R$.
\`E lecito ipotizzare che il numero $M$ di nodi (incroci) della rete cresca proporzionalmente alla superficie della citt\`a, quindi al quadrato del raggio $M\propto R^2$.
Si definisca ora la variabile costo $C$ della rete, la quale dipender\`a sia dal numero di nodi che dalla lunghezza di scala della stessa, ossia $C\propto MR^2$.
Unendo le due relazioni precedenti si ottiene $C\propto M^\frac{3}{2}$ ed \`e possibile constatare che per una citt\`a ideale come quella considerata il costo della logistica stradale cresca all'aumentare del numero di nodi.
La nascita e lo sviluppo delle metropoli, avvenuta nell'ultimo secolo, ha quindi cause da ricercarsi non nel miglioramento della logistica quanto a cambiamenti nella mobilit\`a, come l'introduzione di mezzi pubblici quali autobus, tram e metropolitane.

\end{document}