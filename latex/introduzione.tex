\documentclass[../main.tex]{subfiles}

\begin{document}


\section{Random Walk su network}
In generale, un network è descritto da una matrice di adiacenza $A_{ij}=\left\{0,1\right\}$ in cui la cella $(i,j)$ assume il valore $1$ se il nodo $i$ è connesso al nodo $j$, $0$ altrimenti.
Nei casi considerati in questo studio si assume sempre che $A_{ij}=A_{ji}$ (proprietà di simmetria).
\\Dalla matrice di adiacenza si ricava la matrice dei gradi come
\begin{equation*}
    D_{ij}=
    \begin{cases}
        \sum_jA{ij} \quad &i=j\\
        0 &i\neq j
    \end{cases}
\end{equation*}
che indica il numero delle connessioni per ogni nodo.
\\Una volta note le matrici descritte in precedenza è possibile definire la matrice Laplaciana del network come
\begin{equation}
    L=D-A
\end{equation}
che ha le seguenti proprietà:
\begin{itemize}
    \item è semi-definita positiva;
    \item $L_{ij}>0\Longleftrightarrow i=j$;
    \item $\sum_jL_{ij}=\sum_iL_{ij}=0$, quindi esiste un autovalore nullo $\lambda_0$ con corrispondente autovettore $\vec{v}_0=(1,\ldots,1)$;
    \item $\sum_iL_{ii}=2m$, dove $m$ è il numero totale dei link.
\end{itemize}
Si assuma ora che la rete abbia in totale $M$ nodi e che ognuno di essi possa scambiare particelle coi suoi vicini.
Sia $\pi_{ij}$ la matrice stocastica che definisce la probabilità che una particella effettui il viaggio tra nodi $j\to i$.
Questa possiede le seguenti proprietà:
\begin{itemize}
    \item $A_{ij}=0 \Longrightarrow \pi_{ij}=0$;
    \item $\sum_j\pi_{ij}=1$.
\end{itemize}
Assumendo inoltre di avere $N$ particelle nella rete, è possibile definire la funzione $\delta_\alpha(i,t)$ che vale $1$ se la particella $\alpha$ si trova nel nodo $i$ al tempo $t$, 0 altrimenti.
\\Ogni particella segue quindi la dinamica
\begin{equation}
    \delta_\alpha(i,t+\Delta t)=\sum_j\xi_{ij}^\alpha\delta_\alpha(j,t)
\end{equation}
dove $\xi_{ij}^\alpha$ è una matrice random che prende valori della base standard $\widehat{e}_i\in \mathbb{R}^M$ con probabilità $\pi_{ij}$.
Il numero di particelle nel nodo $i$ al tempo $t$ è dato da
\begin{equation}
    n_i(t)=\sum_\alpha\delta_\alpha(i,t)
\end{equation}
ed è possibile dimostrare \cite{RandomWalks} che la seguente equazione è un integrale del moto
\begin{equation}
    \sum_in_i(t)=N
\end{equation}

\section{Modelli di traffico}

\end{document}