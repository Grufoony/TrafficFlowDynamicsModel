\documentclass[../main.tex]{subfiles}

\begin{document}
Le congestioni nel traffico sono oggi uno dei maggiori problemi per lo sviluppo delle citt\`a.
Queste dipendono strettamente dalla logistica della stessa.
Considerando una citt\`a fittizia, la cui superficie sia approssimabile a quella di una circonferenza di raggio $R$, \`e facile osservare come il numero $M$ di nodi della sua rete stradale debba cresecere proporzionalmente alla superficie, ossia $M\propto R^2$.
Ci si aspetta inoltre che il costo $C$ della rete sia legato sia al numero di nodi, sia alla dimensione della rete, ottenendo $C\propto ML$.
Unendo le relazioni precedenti si dimostra come il costo di una rete stradale sia $C\propto M^\frac{3}{2}$, quindi destinato a crescere all'aumentare del numero di nodi.
Nonostante la considerazione precedente le città sono comunque riuscite a espandersi nel tempo grazie a cambiamenti nella mobilit\`a, come ad esempio l'introduzione di mezzi pubblici quali autobus, tram e metropolitane.
\\Tuttavia, gran parte della mobiltà cittadina si presenta ancora come automobili private le quali, soprattutto negli orari di punta, tendono a saturare la rete e comportano enormi sprechi di tempo e denaro, oltre a causare inquinamento.


\section{Random Walk su network}
In generale, un network \`e descritto da una matrice di adiacenza $\mathcal{A}_{ij}=\left\{0,1\right\}$ in cui la cella $(i,j)$ assume il valore $1$ se il nodo $i$ \`e connesso al nodo $j$, $0$ altrimenti.
Nei casi considerati in questo studio si assume sempre che $\mathcal{A}_{ij}=\mathcal{A}_{ji}$ (proprietà di simmetria).
\\Dalla matrice di adiacenza si pu\`o definire il grado del nodo $i$-esimo come
\begin{equation*}
    d_i=\sum_j\mathcal{A}_{ij}\\
\end{equation*}
che indica il numero delle connessioni per ogni nodo.
\\Una volta note le matrici descritte in precedenza \`e possibile definire la matrice Laplaciana del network come
\begin{equation}
    \mathcal{L}_{ij}=d_i\delta_{ij}-\mathcal{A}_{ij}
\end{equation}
che ha le seguenti propriet\`a:
\begin{itemize}
    \item \`e semi-definita positiva;
    \item $\mathcal{L}_{ij}>0\Longleftrightarrow i=j$;
    \item $\sum_j\mathcal{L}_{ij}=\sum_i\mathcal{L}_{ij}=0$, quindi esiste un autovalore nullo $\lambda_0$ con corrispondente autovettore $\vec{v}_0=(1,\ldots,1)$;
    \item $\sum_i\mathcal{L}_{ii}=2m$, dove $m$ \`e il numero totale dei link.
\end{itemize}
Si assuma ora che la rete abbia in totale $M$ nodi e che ognuno di essi possa scambiare particelle coi suoi vicini.
Sia $\pi_{ij}$ la matrice stocastica che definisce la probabilità che una particella effettui il viaggio tra nodi $j\to i$.
Questa possiede le seguenti proprietà:
\begin{itemize}
    \item $\mathcal{A}_{ij}=0 \Longrightarrow \pi_{ij}=0$;
    \item $\sum_j\pi_{ij}=1$.
\end{itemize}
Assumendo inoltre di avere $N$ particelle nella rete, \`e possibile definire la funzione $\delta_\alpha(i,t)$ che vale $1$ se la particella $\alpha$ si trova nel nodo $i$ al tempo $t$, 0 altrimenti.
\\Ogni particella segue quindi la dinamica
\begin{equation}
    \delta_\alpha(i,t+\Delta t)=\sum_j\xi_{ij}^\alpha\delta_\alpha(j,t)
\end{equation}
dove $\xi_{ij}^\alpha$ \`e una matrice random che prende valori della base standard $\widehat{e}_i\in \mathbb{R}^M$ con probabilità $\pi_{ij}$.
Il numero di particelle nel nodo $i$ al tempo $t$ \`e dato da
\begin{equation}
    n_i(t)=\sum_\alpha\delta_\alpha(i,t)
\end{equation}
ed \`e possibile dimostrare \cite{RandomWalks} che la seguente equazione \`e un integrale del moto
\begin{equation}
    \sum_in_i(t)=N
\end{equation}

\end{document}