\documentclass[../main.tex]{subfiles}

\begin{document}
Le congestioni nel traffico sono ad oggi uno dei maggiori problemi per lo sviluppo delle citt\`a.
Nel solo stato della Florida, ad esempio, \`e stato stimato che dal 2003 al 2007 queste abbiano causato perdite dai 4.5 ai 7 miliardi di dollari annui \cite{florida}.
Oltre a danni economici, le congestioni nel traffico causano anche problemi sociali, come l'aumento del livello di stress negli automobilisti \cite{hennessy1999traffic} e ambientali, in quanto le emissioni da esse causate sono altamente inquinanti e si ripercuotono sulla salute dei cittadini \cite{zhang2013air}.
Nell'ultimo decennio, con l'obiettivo di attenuare questa problematica, nelle grandi citt\`a hanno cominciato a proliferare diverse compagnie di ride-hailing, come Uber.
Nonostante il beneficio economico portato da esse e dalla competitivit\`a del mercato la presenza in gran numero di questi fornitori di servizi va a peggiorare la qualit\`a della mobilit\`a urbana.
Questo effetto diventa abbastanza rilevante in zone nelle quali vi \`e poca domanda o la velocit\`a stradale media \`e sufficientemente bassa.
Ogni operatore aggiunto di questo settore causa, ad esempio, un aumento del numero totale di veicoli su strada del 2.5\% a Manhattan e del 37\% a San Francisco \cite{Kondor2022}.
Inoltre, la presenza di molteplici compagnie di ride-hailing aggiunge ogni anno 9.17 miliardi di km di strade nelle aree metropolitane di Boston, Chicago, Los Angeles, Miami, New York, Philadelphia, San Francisco, Seattle e Washington DC \cite{schaller2018new}.

Un ruolo fondamentale nella gestione delle congestioni nelle citt\`a lo gioca la logistica delle stesse.
Si consideri ora la rete stradale di una citt\`a generica di superfice approssimabile a quella di una circonferenza di raggio $R$.
\`E lecito ipotizzare che il numero $M$ di nodi (incroci) della rete cresca proporzionalmente alla superficie della citt\`a, quindi al quadrato del raggio $M\propto R^2$.
Si definisca ora la variabile costo $C$ della rete, la quale dipender\`a sia dal numero di nodi che dalla lunghezza di scala della stessa, ossia $C\propto MR^2$.
Unendo le due relazioni precedenti si ottiene $C\propto M^\frac{3}{2}$ ed \`e possibile constatare che per una citt\`a ideale come quella considerata il costo della logistica stradale cresca all'aumentare del numero di nodi.
La nascita e lo sviluppo delle metropoli, avvenuta nell'ultimo secolo, ha quindi cause da ricercarsi non nel miglioramento della logistica quanto a cambiamenti nella mobilit\`a, come l'introduzione di mezzi pubblici quali autobus, tram e metropolitane.

Per studiare questo fenomeno si sono evoluti negli anni diversi modelli, basati sia su approcci di tipo microscopico che macroscopico.
Quale sia l'approccio migliore dipende, tuttavia, dal tipo di fenomeno di traffico che si sta studiando.
Un punto di svolta sulla questione si ebbe nel 1959, quando da un modello microscopico basato sul \emph{car-following} emerse una relazione macroscopica equivalente ad una transizione di fase \cite{gazis2002origins}.
Grazie a studi successivi si \`e poi arrivati ad una formulazione lagrangiana dei modelli di \emph{car-following}, dipendente dalla regole assunte.

Scopo di questo lavoro \`e fornire un modello elementare di dinamica microscopica su network che riesca comunque a riprodurre i fenomeni macroscopici caratteristici dei flussi di traffico.
In particolare, lo studio verte sull'analisi dei tre osservabili macroscopici principali, quali velocit\`a, densit\`a e flusso, sia in situazioni di equilibrio (traffico libero e congestioni), sia in transizioni di fase (ciclo di isteresi nel traffico).

\end{document}