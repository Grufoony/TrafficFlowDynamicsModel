\documentclass[12pt,a4paper]{report}
\usepackage[italian]{babel}
\usepackage{newlfont}
\usepackage{color}
\textwidth=450pt\oddsidemargin=0pt
\usepackage{hyperref}
\usepackage{float}
\floatplacement{figure}{H}
\usepackage{tikz}
\usepackage{pgfplots}
\usepgfplotslibrary{groupplots}
\pgfplotsset{compat=1.18}
\usepackage[style=ieee]{biblatex}
\addbibresource{./latex/bibliography.bib}
\usepackage{amsmath, amsthm, amssymb, amsfonts}
\usepackage{subfiles}
\usepackage{graphicx}
\usepackage{csquotes}


\begin{document}
\begin{titlepage}
\begin{center}
{{\Large{\textsc{Alma Mater Studiorum $\cdot$ Universit\`a di Bologna}}}} 
\rule[0.1cm]{15.8cm}{0.1mm}
\rule[0.5cm]{15.8cm}{0.6mm}
\\\vspace{3mm}

{\small{\bf Scuola di Scienze \\ 
Dipartimento di Fisica e Astronomia\\
Corso di Laurea in Fisica}}

\end{center}

\vspace{23mm}

\begin{center}\textcolor{black}{
{\LARGE{\bf Modelli di traffico per la formazione della congestione su una rete stradale}}\\
}\end{center}

\vspace{50mm} \par \noindent

\begin{minipage}[t]{0.47\textwidth}
{\large{\bf Relatore: \vspace{2mm}\\\textcolor{black}{
Prof. Armando Bazzani}\\\\
}}\end{minipage}
%
\hfill
%
\begin{minipage}[t]{0.47\textwidth}\raggedleft \textcolor{black}{
{\large{\bf Presentata da:
\vspace{2mm}\\
Gregorio Berselli}}}
\end{minipage}

\vspace{40mm}

\begin{center}
Anno Accademico \textcolor{black}{2021/2022}
\end{center}

\end{titlepage}
\shipout\null

\thispagestyle{empty}
\addtocounter{page}{-1}
\pagebreak
\hspace{0pt}
\vfill
\begin{flushright}
\emph{Le tangenziali sono soluzioni che permettono a certuni\\
di sfrecciare molto rapidamente da un punto A a un punto B,\\
nel mentre certi altri sfrecciano molto rapidamente dal punto B al punto A.\\
La gente che abita nel punto C, a met\`a strada tra A e B,\\
spesso si chiede cosa ci sia di cos\`i importante nel punto A\\
da indurre tanta gente a correrci spostandosi da B,\\
e cosa ci sia di così importante nel punto B,\\
da indurre tanta gente a correrci spostandosi da A.\\
Cos\`i, le gente del punto C finisce per augurarsi\\
che tutti quei corridori si decidano una buona volta\\
a scegliere una dannata dimora definitiva.}\\
\vspace{\baselineskip}
\emph{Douglas Adams - Guida galattica per gli autostoppisti}
\end{flushright}
\vfill
\hspace{0pt}
\pagebreak

\shipout\null
\chapter*{\centering \Large Abstract}

A partire dagli anni '50 furono sviluppati numerosi modelli con l'intento di studiare i fenomeni connessi al traffico.
Alcuni di questi modelli riuscirono non solo a spiegare i fenomeni per i quali erano stati ideati ma misero in evidenza altre caratteristiche tipiche dei sistemi dinamici, come la presenza di cicli di isteresi e cambiamenti nella distribuzione dei tempi di percorrenza in situazioni di congestione.\\
Questo lavoro si propone di verificare la validit\`a di un modello semplificato ideato per evidenziare i comportamenti tipici di un sistema di traffico, in particolare le congestioni che si vengono a creare sulla rete stradale.
Tale modello viene implementato per mezzo della libreria C++ \emph{Traffic Flow Dynamics Model} reperibile al link \url{https://github.com/Grufoony/TrafficFlowDynaicModel}.
Ai fini dello studio vengono utilizzati i Diagrammi Fondamentali Macroscopici, particolari diagrammi che mettono in relazione gli osservabili principali di un network stradale quali velocit\`a, densit\`a e flusso.
Variando il carico immesso nella rete stradale \`e stato possibile studiare il sistema in diversi regimi: carico costante, carico piccato e carico periodico.
Mediante questi studi sono emerse diverse propriet\`a tipiche di ogni regime e, per alcuni di essi, \`e stata verificate e giustificate la presenza di uno o pi\`u cicli di isteresi.
In ultimo \`e stata effettuata una breve analisi ad-hoc volta a evidenziare i cambiamenti nella distrubuzione dei tempi di percorrenza in relazione al regime di traffico considerato.

\tableofcontents
\listoffigures

\chapter*{Introduzione}
\addcontentsline{toc}{chapter}{Introduzione}
\subfile{latex/introduzione}

\chapter{Fisica del traffico}
\subfile{latex/fisica}

\chapter{Modelli di traffico}
\subfile{latex/teoria}

\chapter{Costruzione del modello}
\subfile{latex/modello}

\chapter{Risultati}
\subfile{latex/risultati}

\appendix

\chapter{Esempi di matrici Laplaciane}
\label{appendix:laplacian}
\subfile{latex/matrici_laplaciane}

\printbibliography
\end{document}
