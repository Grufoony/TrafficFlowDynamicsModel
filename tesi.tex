\documentclass[12pt,a4paper]{report}
\usepackage[italian]{babel}
\usepackage{newlfont}
\usepackage{color}
\textwidth=450pt\oddsidemargin=0pt
\usepackage{float}
\floatplacement{figure}{H}
\usepackage{tikz}
\usepackage{pgfplots}


\begin{document}
\begin{titlepage}
%
%
% UNA VOLTA FATTE LE DOVUTE MODIFICHE SOSTITUIRE "RED" CON "BLACK" NEI COMANDI \textcolor
%
%
\begin{center}
{{\Large{\textsc{Alma Mater Studiorum $\cdot$ Universit\`a di Bologna}}}} 
\rule[0.1cm]{15.8cm}{0.1mm}
\rule[0.5cm]{15.8cm}{0.6mm}
\\\vspace{3mm}

{\small{\bf Scuola di Scienze \\ 
Dipartimento di Fisica e Astronomia\\
Corso di Laurea in Fisica}}

\end{center}

\vspace{23mm}

\begin{center}\textcolor{red}{
%
% INSERIRE IL TITOLO DELLA TESI
%
{\LARGE{\bf TITOLO TESI}}\\
}\end{center}

\vspace{50mm} \par \noindent

\begin{minipage}[t]{0.47\textwidth}
%
% INSERIRE IL NOME DEL RELATORE CON IL RELATIVO TITOLO DI DOTTORE O PROFESSORE
%
{\large{\bf Relatore: \vspace{2mm}\\\textcolor{black}{
Prof. Armando Bazzani}\\\\
%
% INSERIRE IL NOME DEL CORRELATORE CON IL RELATIVO TITOLO DI DOTTORE O PROFESSORE
%
% SE NON AVETE UN CORRELATORE CANCELLATE LE PROSSIME 3 RIGHE
%
\textcolor{red}{
\bf Correlatore: (eventuale)
\vspace{2mm}\\
Prof./Dott. Nome Cognome\\\\}}}
\end{minipage}
%
\hfill
%
\begin{minipage}[t]{0.47\textwidth}\raggedleft \textcolor{black}{
{\large{\bf Presentata da:
\vspace{2mm}\\
Gregorio Berselli}}}
\end{minipage}

\vspace{40mm}

\begin{center}
%
% INSERIRE L'ANNO ACCADEMICO
%
Anno Accademico \textcolor{red}{2021/2022}
\end{center}

\end{titlepage}

\tableofcontents
\listoffigures

\chapter*{Introduzione}
\addcontentsline{toc}{chapter}{Introduzione}

\begin{figure}[H]
    \centering
    \begin{tikzpicture}
        \begin{axis}[ybar interval,
            area style,
            ymin=0,
            xmin=0.24,
            xmax=0.95,
            width = \textwidth,
            height = 0.75\textwidth,
            xlabel = {Velocità},
            ylabel = {Numero di veicoli},]
        \addplot+[
            ybar interval,
            mark=no,
            line width = 1.25pt
        ] file {./data/25.dat};
        \end{axis}
    \end{tikzpicture}
    \caption{\emph{Prova.}}
\end{figure}
\begin{figure}[H]
    \centering
    \begin{tikzpicture}
        \begin{axis}[ybar interval,
            area style,
            ymin=0,
            xmin=0.24,
            xmax=0.95,
            width = \textwidth,
            height = 0.75\textwidth,
            xlabel = {Velocità},
            ylabel = {Numero di veicoli},]
        \addplot+[
            ybar interval,
            mark=no,
            line width = 1.25pt
        ] file {./data/190.dat};
        \end{axis}
    \end{tikzpicture}
    \caption{\emph{Prova.}}
\end{figure}

\section*{Random Walk su network}

\section*{Modelli di traffico}

\chapter{Costruzione del modello}

\chapter{Implementazione}

\chapter{Risultati}

\end{document}
